\documentclass[12pt]{article}
\usepackage{ctex}
\usepackage{graphicx}
\usepackage{indentfirst}
\usepackage{amsmath}
\usepackage{float}
\usepackage{amssymb}
\usepackage{bm}
\title{第十周作业报告}
\author{佐藤拓未 20300186002}
\date{}




\begin{document}
	\maketitle
	
	\begin{center}
		\textbf{第一问}
	\end{center}
\noindent 在区域$\Omega=[0,1]^2$用五点差分格式求解如下问题$$-\Delta u +u=f,$$
\noindent 且$u\vert_{\partial \Omega}=1$. 设A为对应$-\Delta$算子的离散矩阵, 说明:\\
\quad \quad  (1) 观察矩阵$\boldsymbol{A+I}$性质:\\
\quad \quad (2) 格式是否二阶收敛?\\
\quad \quad (3) 求矩阵$\boldsymbol{A}$的特征值、 特征向量.\\
\quad \\
\noindent \textbf{证明:} \\
(1): 为了方便讨论, 我们令$h_x=h_y=\frac{1}{N}$, 而注意到$u\vert _{\partial \Omega}=1$, 因此需要对$f$离散化向量$\overset{\rightarrow}{f_{i,j}}$做一定的修正, 其中$$\overset{\rightarrow}{f_{i,j}}=(u_{1,1},u_{2,1},\cdots,u_{N-1,1},u_{1,2},u_{2,2},\cdots,u_{N-1,2},\cdots,u_{1,N-1},\cdots,u_{N-1,N-1})^T$$
首先由五点差分格式: $-\Delta_h u_{i,j}=\frac{1}{h^2}(-u_{i+1,j}-u_{i-1,j}-u_{i,j+1}-u_{i,j-1}+4u_{i,j})=f_{i,j}$, 可以知道当$i,j=2,3,\cdots,N-2$时, 不需要对$f_{i,j}$作出修改. 而当$i=1,N-1$或$j=1,N-1$时, 由于格式涉及到了边界值, 从而需要对$f_{i,j}$做出修正: 
\begin{align*}
	\widetilde{f_{i,j}}=\begin{cases}
		f_{i,j}+\frac{2}{h^2}, \quad i=j=1,N-1\\
		f_{i,j}+\frac{1}{h^2}, \quad i=1,N-1\quad j=2\\
		f_{i,j}+\frac{1}{h^2}, \quad j=1,N-1\quad i=2
	\end{cases}
\end{align*}
\noindent 因此$u$的边值并未影响五点差分格式矩阵$\boldsymbol{A}$.\\
\noindent 此时
\begin{align*}
	&A=\begin{pmatrix}
		\frac{4}{h^2} & -\frac{1}{h^2} &&&&-\frac{1}{h^2}\\
		-\frac{1}{h^2} &\frac{4}{h^2} & -\frac{1}{h^2}&&&&-\frac{1}{h^2}\\
		& \ddots & \ddots & \ddots&&&&\ddots\\
		&&-\frac{1}{h^2} &\frac{4}{h^2} & -\frac{1}{h^2}&&&&-\frac{1}{h^2}\\
		&&&-\frac{1}{h^2} &\frac{4}{h^2} &&&&&\ddots\\
		-\frac{1}{h^2}&&&&&\frac{4}{h^2}&-\frac{1}{h^2}\\
		&-\frac{1}{h^2}&&&&-\frac{1}{h^2} & \frac{4}{h^2} &-\frac{1}{h^2}\\
		&&\ddots &&&&\ddots&\ddots&\ddots
	\end{pmatrix}_{(N-1)^2 \times (N-1)^2}
\end{align*}
\noindent 而易知$\boldsymbol{A+I}$又是严格对角占优矩阵, 从而可知矩阵$\boldsymbol{A+I}$是非异的.\\
\quad \\
(2) 首先考虑其局部截断误差$R_{i,j}=-\Delta_hu(x_i,y_j)+u(x_i,y_j)-f(x_i,y_j)$, 可知由二维Taylor展开有
\begin{align*}
	R_{i,j}=&-\frac{1}{h^2} \cdot [h^2(\frac{{\rm \partial}^2u(x_i,y_j)}{{\rm \partial}x^2}+(\frac{{\rm \partial}^2u(x_i,y_j)}{{\rm \partial}y^2})+\frac{h^4}{12}(\frac{{\rm \partial}^4u(x_i,y_j)}{{\rm \partial}x^4}+\frac{{\rm \partial}^4u(x_i,y_j)}{{\rm \partial}y^4})+O(h^6)]\\
	& +u(x_i,y_j) - (-\Delta u(x_i,y_j)+u(x_i,y_j))\\
	=&-\frac{h^2}{12}(\frac{{\rm \partial}^4u(x_i,y_j)}{{\rm \partial}x^4}+\frac{{\rm \partial}^4u(x_i,y_j)}{{\rm \partial}y^4})+O(h^4)
\end{align*}
\noindent 再考虑$e_{i,j}=u(x_i,y_j)-u_{i,j}$, 则由$-\Delta_hu(x_i,y_j)+u(x_i,y_j)=R_{i,j}+f(x_i,y_j)$, 并且总是有$f(x_i,y_j)=f_{i,j}$. 同时, 在边界上有$e_{i,j}\vert_{\partial\Omega _h}=0$.\\
最后考虑如下离散化边值问题
\begin{align*}
	\begin{cases}
		&-\Delta_he_{i,j}+e_{i,j}=R_{i,j}\\
		&e_{i,j}\vert_{\partial\Omega _h}=0
	\end{cases}
\end{align*}
\noindent 利用比较原理: 令$v_{i,j}=e_{i,j}-\vert R_{i,j}\vert_{ \Omega_h} $, 其中$\vert R_{i,j}\vert_{ \Omega_h}$是$R_{i,j}$在$\Omega$离散化区域$\Omega_h$上的绝对值最大值, 其满足
\begin{align*}
	\begin{cases}
		&-\Delta_hv_{i,j}+v_{i,j}\le0\\
		&v_{i,j}\vert _{\partial\Omega_h}\le0
	\end{cases}
\end{align*}
\noindent 由极值原理可得$e_{i,j}\le \vert R_{i,j}\vert_{ \Omega_h}$.\\
同理, 令$w_{i,j}=-e_{i,j}-\vert R_{i,j}\vert_{ \Omega_h}$, 可得$-\vert R_{i,j}\vert_{ \Omega_h}\le e_{i,j}$, 从而综上所述可得$\vert e_{i,j}\vert \le\vert R_{i,j}\vert_{ \Omega_h} $, 所以全局的误差$\vert e_{i,j}\vert = O(h^2), \forall i,j=1,\cdots,N-1$.\\
\noindent 最后记$\overset{\rightarrowtail}{R}=(R_{i,j})^T$为一个列向量, $\vert \vert\overset{\rightarrowtail}{R}\vert \vert_{l^2}=(\sum_{i,j=1}^{N-1}(R_{i,j})^2)^{\frac{1}{2}}\le (N^2O(h^4))\frac{1}{2}=O(h) $, 故$h\cdot \vert \vert\overset{\rightarrowtail}{R}\vert \vert_{l^2}=O(h^2)$.\\
\quad \\
(3): 为了考虑矩阵$\boldsymbol{A}$的特征值与特征向量, 将其写成$\boldsymbol{A=A_y\otimes I_x+I_y \otimes A_x}$, 并且由于各自$\boldsymbol{A_x},\boldsymbol{A_y}$为对称矩阵, 从而可以对角化, 记$\boldsymbol{A_x}=\boldsymbol{Q_x\Lambda_xQ_x}^T$,$\boldsymbol{A_y}=\boldsymbol{Q_y\Lambda_yQ_y}^T$, 最终可得
\begin{align*}
	\boldsymbol{A}&=(\boldsymbol{Q_y\otimes Q_x})(\boldsymbol{\Lambda_y \otimes I_x}+\boldsymbol{I_y \otimes \Lambda_x})(\boldsymbol{Q_y\otimes Q_x})^T
\end{align*}
\noindent 从而$\boldsymbol{A}$的特征值就形如$\lambda_{j,k}=\frac{4}{h^2}(\sin^2(\frac{j \pi h}{2})+\sin^2(\frac{k\pi h}{2})),\forall i,j=1,\cdots,N-1$, 即两个三点差分格式矩阵所对应的特征值相加. 由上式也能看出$\boldsymbol{A}$的特征向量就是$\boldsymbol{Q_y \otimes Q_x}$的各列, 即各分量$\overset{\rightarrow}{u_{j,k}}^{(m,l)}=\sqrt{\frac{2}{N}}\sin(j\pi x_m)\cdot \sqrt{\frac{2}{N}}\sin(k\pi y_l)=\frac{2}{N}sin(\frac{j \pi m}{N})sin(\frac{k \pi l}{N})$, 此分量代表的是对应于特征值$\lambda_{j,k}$的特征向量第$m(N-1)+l$分量.
\end{document}