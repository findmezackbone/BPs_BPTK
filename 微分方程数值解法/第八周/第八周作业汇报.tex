\documentclass[12pt]{article}
\usepackage{ctex}
\usepackage{graphicx}
\usepackage{indentfirst}
\usepackage{amsmath}
\usepackage{float}
\usepackage{amssymb}
\usepackage{bm}
\title{第八周作业报告}
\author{佐藤拓未 20300186002}
\date{}

\begin{document}
	
	
	
	
	\maketitle
	
	
	\begin{center}
		\textbf{第一问}
	\end{center}

\noindent 对于一般的椭圆型方程, $Lu = -\nabla\cdot (A\nabla u) + (b\cdot\nabla)u + cu = f \in\Omega\subset R^d$, 类似书中引理3.1.4的条件: $c-\frac{1}{2}div (b)\geq c_0 >0$, 证明相应的Lax-Milgram引理.\\
\quad \\
\noindent \textbf{证明:} 由$(f,u)=(Lu,u)$, 可知
\begin{align*}
	(f,u)&=(-\nabla\cdot (A\nabla u),u)+((b\cdot\nabla)u,u)+(cu,u)
\end{align*}

\noindent 先考虑$(-\nabla\cdot (A\nabla u),u)$, 可知
\begin{align*}
	(-\nabla\cdot (A\nabla u),u)&=\int_{\Omega}^{}-\nabla\cdot (A\nabla u)u{\rm d}x\\
	&=\int_{\Omega}^{}-\nabla\cdot (uA\nabla u){\rm d}x+\int_{\Omega}^{}(A\nabla u)\cdot\nabla u{\rm d}x
\end{align*}
\noindent 由于$\int_{\Omega}^{}\nabla\cdot (u(A\nabla u)){\rm d}x=\int_{\partial \Omega}^{}u(A\nabla u)\cdot\overset{\rightarrow}{n_x}{\rm d}S_x$, 以及$u\vert _{\partial \Omega}=0$, 从而$\int_{\Omega}^{}\nabla\cdot (u(A\nabla u)){\rm d}x=0$.\\
为了对等式右边第二项进行下有界控制, 需要$A$的\underline{强制性}: 即$A$满足$(Au,u)\geq \alpha_0(u,u)$, 其中$\alpha_0 >0$. 那么, 此时就有
\begin{align*}
	(-\nabla\cdot (A\nabla u),u)&=\int_{\Omega}^{}-\nabla\cdot (A\nabla u)u{\rm d}x  \\
	&\ge \alpha_0 \int_{\Omega}^{}\vert \nabla u\vert ^2 {\rm d}x
\end{align*}
\noindent 再考虑$((b\cdot\nabla)u,u)$
\begin{align*}
	2\int_{\Omega}^{}u(b\cdot\nabla)u{\rm d}x&=\int_{\Omega}^{}\nabla \cdot (u^2b){\rm d}x-\int_{\Omega}^{}u^2div(b){\rm d}x
\end{align*}
\noindent 类似地, $\int_{\Omega}^{}\nabla \cdot (u^2b){\rm d}x=\int_{\partial \Omega}^{} u^2b\cdot\overset{\rightarrow}{n_x}{\rm d}S_x\overset{u\vert _{\partial \Omega}=0}{=}0$, 从而$((b\cdot\nabla)u,u)=-\frac{1}{2}\int_{\Omega}^{}u^2div(b){\rm d}x$. 那么有以下
\begin{align*}
	(f,u)&\ge\alpha_0 \int_{\Omega}^{}\vert \nabla u\vert ^2 {\rm d}x+\int_{\Omega}^{}(c-\frac{1}{2}div(b))u^2{\rm d}x\\
	&\ge\alpha_0 \int_{\Omega}^{}\vert \nabla u\vert ^2 {\rm d}x+\int_{\Omega}^{}c_0u^2{\rm d}x\\
	&\ge \alpha \vert \vert u \vert \vert^2_{H^1(\Omega)}
\end{align*}
\noindent 其中$\alpha = min\{\alpha_0,c_0\}$
\noindent 最终由Hölder不等式: $(f,u)\le \vert \vert f \vert \vert_{L^2(\Omega)}\vert \vert u\vert \vert_{L^2(\Omega)}\le\vert \vert f \vert \vert_{L^2(\Omega)}\vert \vert u\vert \vert_{H^1(\Omega)}$, 可得
\begin{align*}
	\alpha \vert \vert u\vert \vert_{H^1(\Omega)}\le\vert \vert f \vert \vert_{L^2(\Omega)}
\end{align*}
\quad \\

\begin{center}
	\textbf{第二问}
\end{center}
\noindent 给出下列Poisson方程两点边值问题的Green函数
\begin{align*}
	\begin{cases}
		&-\frac{{\rm d^2}u}{{\rm d}x^2} = f, \quad x \in(0,1)\\
		&u(0)=u^{'}(1)=0
	\end{cases}
\end{align*}
\quad \\
\noindent \textbf{解: }设上述两点边值问题的Green函数$G$具有以下分段形式
\begin{align*}
	G(x;x_0)=\begin{cases}
		&G_1(x;x_0),\quad 0<x<x_0\\
		&G_2(x;x_0),\quad x_0<x<1
	\end{cases}
\end{align*}
\noindent 首先考虑形式解满足$(u,\Delta{G})=-u(x_0)$, 由Green第二公式或分部积分公式可得
\begin{align*}
	(u,\Delta{G})-(\Delta{u},G)&=\int_{0}^{1}\nabla \cdot(u\nabla G){\rm d}x -\int_{0}^{1}\nabla \cdot(G\nabla u){\rm d}x\\
	-u(x_0)+(f,G)&=u(1)\frac{{\rm d}G}{{\rm d}x}(1;x_0)+u^{'}(0)G(0;x_0)
\end{align*}
\noindent 若有$G_1(0;x_0)=\frac{{\rm d}G}{{\rm d}x}(1;x_0)=0$, 则
\begin{align*}
	u(x_0)&=(f,G)\\
	&=\int_{0}^{x_0}f(x)G_1(x;x_0){\rm d}x +\int_{x_0}^{1}f(x)G_2(x;x_0){\rm d}x
\end{align*}
\noindent 上式两边对参变量$x_0$求导, 则
\begin{align*}
	u^{'}(x_0)&=f(x_0)G_1(x_0;x_0)+\int_{0}^{x_0}f(x)\frac{{\rm \partial}G_1}{{\rm \partial}x_0}(x;x_0){\rm d}x\\
	&\quad -f(x_0)G_2(x_0;x_0)+\int_{x_0}^{1}f(x)\frac{{\rm \partial}G_2}{{\rm \partial}x_0}(x;x_0){\rm d}x
\end{align*}
\noindent 此时又若$G_1,G_2$满足: $\frac{{\rm \partial}G_1}{{\rm \partial}x_0}(x_0;x_0)-\frac{{\rm \partial}G_2}{{\rm \partial}x_0}(x_0;x_0)=-1$, $G_1(x_0;x_0)=G_2(x_0;x_0)$以及$\frac{{\rm \partial}^2G_1}{{\rm \partial}x_0^2}(x;x_0)=\frac{{\rm \partial}^2G_2}{{\rm \partial}x_0^2}(x;x_0)=0$, 那么
\begin{align*}
	u^{''}(x_0)&=f(x_0)(\frac{{\rm \partial}G_1}{{\rm \partial}x_0}(x_0;x_0)-\frac{{\rm \partial}G_2}{{\rm \partial}x_0}(x_0;x_0))\\
	&\quad +\int_{0}^{x_0}f(x)\frac{{\rm \partial}^2G_1}{{\rm \partial}x_0^2}(x;x_0){\rm d}x+\int_{x_0}^{1}f(x)\frac{{\rm \partial}^2G_2}{{\rm \partial}x_0^2}(x;x_0){\rm d}x\\
	&=-f(x_0)
\end{align*}
\noindent 此时定义Green函数$G$为以下
\begin{align*}
	G(x;x_0)\overset{\triangle}{=}\begin{cases}
		&x,\quad 0<x<x_0\\
		&x_0,\quad x_0<x<1
	\end{cases}
\end{align*}
\noindent 则容易得知$G$满足上述所有要求, 此即Poisson方程混合边值问题的Green函数.













\begin{center}
	\textbf{第三问}
\end{center}
\noindent 说明极值原理中的$c(x)$为负时, 极值原理可能不成立. 考虑一维的Helmholtz方程, 找u满足$$\frac{{\rm d}^2 u}{{\rm d}x^2}+k^2u=1,\quad u(0)=u(1)=0.$$

\quad \\
\textbf{证明:} 记$Lu=-\frac{{\rm d}^2 u}{{\rm d}x^2}-k^2u=-1<0$, 并且易知当$k=2\pi$时, $u=\frac{1}{4\pi^2}(1-\cos(2\pi x)-\sin(2\pi x))=\frac{1}{4\pi^2}(1-\sqrt{2}\cos(2\pi x-\frac{\pi}{4}))$是下列边值问题的一个特解
\begin{align*}
	\begin{cases}
		&\frac{{\rm d}^2u}{{\rm d}x^2}+4\pi^2u=1,\quad x\in(0,1)\\
		&u(0)=u(1)=0
	\end{cases}
\end{align*}
\noindent 同时容易知道$\max_{x\in[0,1]}u(x)=\frac{1}{4\pi^2}(1+\sqrt{2})>0, \min_{x\in[0,1]}u(x)=\frac{1}{4\pi^2}(1-\sqrt{2})<0$, 即最大值在区间内部取到, 但不在边界取到, 从而极值原理不成立.\\

\quad \\

\begin{center}
	\textbf{第四问}
\end{center}
\noindent 设$u(0)=u(1)=0$且满足$-\frac{{\rm d}^2u}{{\rm d}x^2}=f(x)$. 利用Green函数表达式(3.1.19), 证明$$u(x)=\int_{0}^{1}G(x;x_0)f(x_0){\rm d}x_0.$$
\noindent \textbf{证明:}记$u(x)=\int_{0}^{1}G(x;y)f(y){\rm d}y$, 要验证其满足以下Poisson方程边值问题
\begin{align*}
	\begin{cases}
		&-\frac{{\rm d}^2u}{{\rm d}x^2}=f(x),\quad  x\in(0,1)\\
		&u(0)=u(1)=0
	\end{cases}
\end{align*}
\noindent 由Green函数定义
\begin{align*}
	G(x;y)=\begin{cases}
		&(1-y)x,\quad x\in(0,y)\\
		&y(1-x),\quad x\in(y,1)
	\end{cases}
\end{align*}
\noindent 可知
\begin{align*}
	u(0)&=\int_{0}^{1}f(y)(1-y)\cdot0\cdot{\rm d}y=0\\
	u(1)&=\int_{0}^{1}f(y)y\cdot0\cdot{\rm d}y=0
\end{align*}
\noindent 而$\forall x\in(0,1)$
\begin{align*}
	u(x)&=\int_{0}^{1}G(x;y)f(y){\rm d}y\\
	&=\int_{0}^{x}f(y)y(1-x){\rm d}y+\int_{x}^{1}f(y)(1-y)x{\rm d}y
\end{align*}
\noindent 对上式两边关于$x$求导, 则有
\begin{align*}
	\frac{{\rm d}u}{{\rm d}x}&=f(x)x(1-x)-\int_{0}^{x}f(y)y{\rm d}y\\
	&\quad -f(x)(1-x)x+\int_{x}^{1}f(y)(1-y){\rm d}y\\
	&=\int_{x}^{1}f(y){\rm d}y-\int_{0}^{1}f(y)y{\rm d}y
\end{align*}
\noindent 从而$-\frac{{\rm d}^2u}{{\rm d}x^2}=f(x)$.
	
	
	
	
	
\end{document}